\chapter{SymmSAT}\label{chap:symmSAT}


\subsubsection{Drawbacks of the static-based approaches.} In the general case,
the size of the \textit{sbp} can be exponential in the number of variables of
the problem so that they cannot be totally computed. Even in more favorable
situations, the size of the generated \textit{sbp} is often too large to be
effectively handled by a SAT solver~\cite{Luks2004}. On the other hand, if
only a subset of the symmetries is considered then the resulting search pruning
will not be that interesting and its effectiveness depends heavily on the
heuristically chosen symmetries \cite{biere2009handbook}. Besides, these approaches
are preprocessors, so their combination with other techniques, such as
\emph{symmetry propagation}~\cite{Devriendt12}, can be very hard. Also, tuning
their parameters during the solving turns out to be very difficult. For all
these reasons, some classes of SAT problems cannot be solved yet despite
exhibiting symmetries.



%
%\emph{static symmetry breaking} has some drawbacks which are
%\begin{itemize}
%	\item pre-generation of the constraints
%	\item maybe unused (larges) clauses by the solver
%	\item 
%\end{itemize}


\subsubsection{Proposed solution. } To handle these issues, we propose a new
approach that reuses the principles of the static approaches, but operates
dynamically: the symmetries are broken during the search process without any
pre-generation of the \textit{sbp}. To do so, we elaborate the notions of
\emph{symmetry status tracking} and \emph{effective symmetric breaking
	predicates} (\emph{esbp}).

The approach is implemented using a couple of components: (1) a
\textit{Conflict Driven Clauses Learning (CDCL) search engine}; (2) \textit{a
	symmetry controller}. Roughly speaking, the first component performs the
classical search activity on the SAT problem, while the second observes the
engine and maintains the status of the symmetries. When the controller detects
a situation where the engine is starting to explore a redundant
part\footnote{Isomorphic to a part that has been/will be explored.}, it orders
the engine to operate a backjump. The detection is performed thanks to
\emph{symmetry status tracking} and the backjump order is given by a simple
injection of an \emph{esbp} computed on the fly.

The main advantage of such an approach is to cope with the heavy (and
potentially blocking) pre-generation phase of the static-based approaches, but
also offers opportunities to combine with other dynamic-based approaches, like
the \textit{symmetry propagation} technique~\cite{Devriendt12}. It also gives
more flexibility for adjusting some parameters on the fly. Moreover, the
overhead for non symmetric formulas is reduced to the computation time of the
graph automorphism.

The extensive evaluation of our approach on the symmetric formulas of the last
six SAT contests shows that it outperforms the state-of-the-art techniques, in
particular on unsatisfiable instances, which are the hardest class of the
problem.



What we propose here is a best effort approach that tries to eliminate,
\textit{dynamically}, the \textit{non lex-leading} assignments with a minimal
computation effort. To do so, we first introduce the notions of
\textit{reducer}, \textit{inactive} and \textit{active} permutation with
respect to an assignment $\alpha$.

\begin{definition}[Reducer, inactive and active permutation] A permutation $g$
	is a \emph{reducer} of an assignment $\alpha$ if $g.\alpha < \alpha$ (hence $\alpha$ cannot be
	the lex-leader of its orbit. $g$ reduces it and all its extensions). $g$ is
	\emph{inactive} on $\alpha$ when $\alpha < g.\alpha$ (so, $g$ cannot reduce $\alpha$ and all
	the extensions). A symmetry is said to be \emph{active} with respect to $\alpha$
	when it is neither inactive nor a reducer of $\alpha$. \end{definition}

Proposition~\ref{prop:status} restates this definition in terms of variables
and is the basis of an efficient algorithm to keep track of the status of a
permutation during the solving. Let us, first, recall that the \emph{support},
$\Vars_g$, of a permutation $g$ is the set $\{ v \in \Vars \mid g(v) \neq v\}$.

\begin{proposition}
	\label{prop:status}
	Let $\alpha \in \Assignments(\Vars)$ be an assignment, $g \in \Group{\Vars}$ a permutation and $ \Vars_g \subseteq  \Vars$ the support of $g$. We say that $g$ is:
	\begin{enumerate}
		\item  \emph{a reducer of} $\alpha$  if there exists a variable $v \in \Vars_g$
		such that:
		\begin{itemize}
			\item $\forall\ v' \in \Vars_g$, s. t. $v' \prec v$, either $\{v', g^{-1}(v')\}\subseteq\alpha $ or $\{\neg v', \neg g^{-1}(v')\} \subseteq \alpha $,
			\item $\{v, \neg g^{-1}(v)\} \subseteq \alpha$;
		\end{itemize}
		\item  \emph{inactive} on $\alpha$  if there exists a variable $v \in \Vars_g$
		such that:
		\begin{itemize}
			\item $\forall\ v' \in \Vars_g$, s. t. $v' \prec v$, either $\{v', g^{-1}(v')\}\subseteq\alpha $ or $\{\neg v', \neg g^{-1}(v')\} \subseteq \alpha $,
			\item $\{\neg v, g^{-1}(v)\} \subseteq \alpha$;
		\end{itemize}
		\item  \emph{active} on $\alpha$, otherwise.
	\end{enumerate}
\end{proposition}

When $g$ is a \textit{reducer} of $\alpha$ we can define a predicate that contradicts $\alpha$ yet preserves the satisfiability of the formula. Such a predicate will be used to discard $\alpha$, and all its extensions, from a further visit and hence pruning the search tree.

\begin{definition}[Effective Symmetry Breaking Predicate]
	\label{def:esbp}
	Let $\alpha \in \Assignments(\Vars)$, and $g \in \Group{\Vars}$.
	We say that the formula $\psi$ is an effective symmetry breaking predicate (\textit{esbp} for short) for $\alpha$ under $g$ if:
	$$\alpha \not\models \psi \text{ and for all }\beta \in \Assignments(\Vars), \beta \not\models \psi \Rightarrow g.\beta < \beta$$
\end{definition}

The next definition gives a way to obtain such an effective symmetry-breaking predicate from an assignment and a reducer.

\begin{definition}[A construction of an \emph{esbp}]
	\label{def:eta}
	Let $\varphi$ be a formula.
	Let $g$ be a symmetry of $\varphi$ that reduces an assignment $\alpha$.
	Let $v$ be the variable whose existence is given by item 1. in Proposition~\ref{prop:status}.
	Let $U = \{ v', \neg v' \mid v' \in \Vars_g \text{ and } v'~\preceq~v\}$.
	We define $\eta(\alpha, g)$ as $(U \cup g^{-1}.U) \setminus \alpha$.
\end{definition}

\textbf{Example}. Let us consider $\Vars = \{x_1, x_2, x_3, x_4, x_5\}$, $g =
(x_1\,x_3)(x_2\,x_4)$, and a partial assignment $\alpha = \{x_1, x_2,
x_3, \neg x_4\}$. Then, $g.\alpha = \{x_1, \neg x_2, x_3, x_4\}$ and $v = x_2$.
So, $U = \{x_1, \neg x_1, x_2, \neg x_2\}$ and $g^{-1}.U = \{x_3, \neg x_3,
x_4, \neg x_4\}$ and we can deduce than $\eta(\alpha, g) = (U \cup g^{-1}.U)
\setminus \alpha = \{\neg x_1, \neg x_2, \neg x_3, x_4\}$.

\begin{proposition}
	\label{prop:eta}
	$\eta(\alpha, g)$ is an effective symmetry-breaking predicate.
\end{proposition}
\begin{proof}
	It is immediate that $\alpha \not\models \eta(\alpha, g)$.
	
	Let $\beta \in \Assignments(\Vars)$ such that $\beta \wedge \eta(\alpha, g)$ is \unsat. We denote a $\alpha'$
	and $\beta'$ as the restrictions of $\alpha$ and $\beta$ to the variables in $\{ v' \in
	\Vars_g \mid v'~\preceq~v \}$. Since $\beta \wedge \eta(\alpha, g)$ is \unsat, $\alpha' = \beta'$.
	But $g.\alpha' < \alpha'$, and $g.\beta' < \beta'$. By monotonicity of $<$, we thus also have
	$g.\beta < \beta$. \end{proof}

\medskip\noindent It is important to observe that the notion of \textit{ebsp}
is a refinement of the classical concept of \textit{sbp} defined in
\cite{aloul06}. In particular, like \textit{sbp}, \textit{esbp} preserve
satisfiability.

\begin{theorem}[Satisfiability preservation]
	Let $\varphi$ be a formula and $\psi$ an \textit{esp} for some assignment $\alpha$ under $g \in S(\varphi)$. Then,
	$$\varphi~and ~\varphi \wedge \psi \text{ are equi-satisfiable}.$$
\end{theorem}

\begin{proof}
	
	If $\varphi \wedge \psi$ is SAT then $\varphi$ is trivially SAT. If
	$\varphi$ is SAT, then there is some assignment $\beta$ that satisfies $\varphi$.
	Without loss of generality, $\beta$ can be chosen to be the lex-leader of its
	orbit under $S(\varphi)$. Thus, $g$ does not reduce $\beta$, which implies that
	$\beta \models \psi$.
	
\end{proof}


\section{Experiments}
