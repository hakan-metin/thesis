\chapter{SymmSAT: Between Static and Dynamic\\ Symmetry Breaking}\label{chap:symmSAT}
\minitoc
This chapter presents our first contribution published in TACAS 2018 conference~\ref{algorithm:cdcl_cosy}. 
\section{General idea}
%\subsubsection{Drawbacks of the static-based approaches.} 
In the static symmetry breaking approach constraints are added to the original 
problem that avoids the solver to visit symmetrical search space.
But, in the general case,
the size of the \textit{sbp} can be exponential in the number of variables of
the problem so that they cannot be entirely computed. Even in more favorable
situations, the size of the generated \textit{sbp} is often too large to be
effectively handled by a SAT solver~\cite{Luks2004}. On the other hand, if
only a subset of the symmetries is considered then the resulting search pruning
will not be that interesting and its effectiveness depends heavily on the
heuristically chosen symmetries \cite{biere2009handbook}. Besides, these approaches
are preprocessors, so their combination with other techniques, such as
\emph{symmetry propagation}~\cite{Devriendt12}, can be very hard. Also, tuning
their parameters during the solving turns out to be tough. For all
these reasons, some classes of SAT problems cannot be solved easily yet despite
the presence of symmetries.
To handle these issues, we propose a new
approach that reuses the principles of the static approaches, but operates
dynamically: the symmetries are broken during the search process without any
pre-generation of the \textit{sbp}. It is a best effort approach that tries to eliminate,
\textit{dynamically}, the \textit{non lex-leading} assignments with a minimal
computation effort. To do so, we first introduce the notions of
\textit{reducer}, \textit{inactive} and \textit{active} permutations (with
respect to an assignment $\alpha$) and \emph{effective symmetric breaking predicates} (\emph{esbp}).
\begin{definition}[Reducer, inactive and active permutation]
  A permutation $g$ is a \emph{reducer} of an assignment $\alpha$ if $g.\alpha < \alpha$ 
  (hence $\alpha$ cannot be the lex-leader of its orbit. The permutation  $g$ reduces the assignment and all its extensions).\\
 The permutation $g$ is \emph{inactive} on $\alpha$ when $\alpha < g.\alpha$ (so $g$ cannot reduce $\alpha$ and all
 its extensions). A symmetry is said to be \emph{active} with respect to $\alpha$
 when it is neither inactive nor a reducer of $\alpha$. 
\end{definition}
Proposition~\ref{prop:status} restates this definition in terms of variables
and is the basis of an efficient algorithm to track the status of a
permutation during the solving. Let us, first, recall that the \emph{support}
 of a permutation $g$, $\support_g$, is the set $\{ v \in \Vars \mid g.v \neq v\}$.
 \clearpage
\begin{proposition}
 \label{prop:status}
 Let $\alpha \in \Assignments(\Vars)$ be an assignment, $g \in \Group$, a permutation and $ \support_g \subseteq  \Vars$ the support of $g$. We say that $g$ is:
 \begin{enumerate}
  \item  \emph{a reducer of} $\alpha$  if there exists a variable $v \in \support_g$
  such that:
  \begin{itemize}
   \item $\forall\ v' \in \support_g$, s. t. $v' \prec v$, either $\{v', g^{-1}(v')\}\subseteq\alpha $ or $\{\neg v', \neg g^{-1}(v')\} \subseteq \alpha $,
   \item $\{v, \neg g^{-1}(v)\} \subseteq \alpha$;
  \end{itemize}
  \item  \emph{inactive} on $\alpha$  if there exists a variable $v \in \support_g$
  such that:
  \begin{itemize}
   \item $\forall\ v' \in \support_g$, s. t. $v' \prec v$, either $\{v', g^{-1}(v')\}\subseteq\alpha $ or $\{\neg v', \neg g^{-1}(v')\} \subseteq \alpha $,
   \item $\{\neg v, g^{-1}(v)\} \subseteq \alpha$;
  \end{itemize}
  \item  \emph{active} on $\alpha$, otherwise.
 \end{enumerate}
\end{proposition}

When $g$ is a \textit{reducer} of $\alpha$ we can define a predicate that contradicts $\alpha$ yet preserves the satisfiability of the formula. Such a predicate will be used to discard $\alpha$, and all its extensions, from a further visit and hence pruning the search tree.
\begin{definition}[Effective Symmetry Breaking Predicate]
 \label{def:esbp}
 Let $\alpha \in \Assignments(\Vars)$, and $g \in \Group_{\Vars}$.
 We say that the formula $\psi$ is an effective symmetry breaking predicate (\textit{esbp} for short) for $\alpha$ under $g$ if:
 $$\alpha \not\models \psi \text{ and for all }\beta \in \Assignments(\Vars), \beta \not\models \psi \Rightarrow g.\beta < \beta$$
\end{definition}
The next definition gives a way to obtain such an effective symmetry-breaking predicate from an assignment and a reducer.
\begin{definition}[A construction of an \emph{esbp}]\label{def:eta}
 Let $\varphi$ be a formula.
 Let $g$ be a symmetry of $\varphi$ that reduces an assignment $\alpha$.
 Let $v$ be the variable whose existence is given by item 1. in Proposition~\ref{prop:status}.
 Let $U = \{ v', \neg v' \mid v' \in \Vars_g \text{ and } v'~\preceq~v\}$.
 We define $\eta(\alpha, g)$ as $(U \cup g.U) \setminus \alpha$.
\end{definition}

\textbf{Example}. Let us consider $\Vars = \{x_1, x_2, x_3, x_4, x_5\}$, $g =
(x_1\,x_3)(x_2\,x_4)$, and a partial assignment $\alpha = \{x_1, x_2,
x_3, \neg x_4\}$. Then, $g.\alpha = \{x_1, \neg x_2, x_3, x_4\}$ and $v = x_2$.
So, $U = \{x_1, \neg x_1, x_2, \neg x_2\}$ and $g^{-1}.U = \{x_3, \neg x_3,
x_4, \neg x_4\}$ and following the Definition \ref{def:eta}, we can deduce than $\eta(\alpha, g) = (U \cup g.U)
\setminus \alpha = \{\neg x_1, \neg x_2, \neg x_3, x_4\}$.
\begin{proposition}
 \label{prop:eta}
 $\eta(\alpha, g)$ is an effective symmetry-breaking predicate.
\end{proposition}
\begin{proof}
 It is immediate that $\alpha \not\models \eta(\alpha, g)$.
 
 Let $\beta \in \Assignments(\Vars)$ such that $\beta \wedge \eta(\alpha, g)$ is \unsat. We denote a $\alpha'$
 and $\beta'$ as the restrictions of $\alpha$ and $\beta$ to the variables in $\{ v' \in
 \Vars_g \mid v'~\preceq~v \}$. Since $\beta \wedge \eta(\alpha, g)$ is \unsat, $\alpha' = \beta'$.
 But $g.\alpha' < \alpha'$, and $g.\beta' < \beta'$. By monotonicity of $<$, we thus also have
 $g.\beta < \beta$. \end{proof}
\medskip\noindent It is important to observe that the notion of \textit{ebsp}
is a refinement of the classical concept of \textit{sbp} defined in \cite{aloul06}. Specifically, like \textit{sbp}, \textit{esbp} preserve satisfiability.

\begin{theorem}[Satisfiability preservation]
 Let $\varphi$ be a formula and $\psi$ an \textit{esbp} for some assignment $\alpha$ under $g \in G_{\varphi}$. Then,
 $$\varphi~and ~\varphi \wedge \psi \text{ are equi-satisfiable}.$$
\end{theorem}
\begin{proof}
 
 If $\varphi \wedge \psi$ is SAT then $\varphi$ is trivially SAT. If
 $\varphi$ is SAT, then there is some assignment $\beta$ that satisfies $\varphi$.
 Without loss of generality, $\beta$ can be chosen to be the lex-leader of its
 orbit under $G_{\varphi}$. Thus, $g$ does not reduce $\beta$, which implies that
 $\beta \models \psi$.
 
\end{proof}

\subsection{Algorithm}
This section describes how to augment the state-of-the-art CDCL algorithm with
the aforementioned concepts to develop an efficient symmetry-guided SAT
solving algorithm. 
The approach is implemented using a couple of components: (1) a
\textit{Conflict Driven Clauses Learning (CDCL) search engine}; (2) \textit{a symmetry controller}. Roughly speaking, the first component performs the
classical search activity on the SAT problem, while the second observes the
engine and maintains the status of the symmetries. When the controller detects
a situation where the engine is starting to explore a redundant
part\footnote{Isomorphic to a part that has been/will be explored.}, it orders
the engine to operate a backjump. The detection is performed thanks to
\emph{symmetry status tracking} and the backjump order is given by a simple
injection of an \emph{esbp} computed on the fly.
%Principle of CDCL is described in section \ref{sec:cdcl}, 
\Cref{algorithm:cdcl_cosy} explains how to extend the CDCL algorithm described in \Cref{sec:cdcl}  with a \emph{symmetry controller} component.% which guides the behavior of CDCL algorithm depending on the status of symmetries.
\begin{algorithm}[!htbp]
	\SetKwProg{Fn}{function}{}{}
	\SetKwData{C}{SymController}
	\SetKwFunction{CDCL}{CDCLSym}
	\SetKwFunction{unitPropagation}{unitPropagation}
	\SetKwFunction{analyzeConflict}{analyzeConflict}
	\SetKwFunction{addLearntClause}{addLearntClause}
	\SetKwFunction{assignNewLiteral}{assignDecisionLiteral}
	\SetKwFunction{backjumpPolicy}{backjumpAndRestartPolicies}
	\SetKwFunction{isNotMinimal}{isNotLexLeader}
	\SetKwFunction{SBP}{generateEsbp}
	\SetKwFunction{notifyAssigned}{updateAssign}
	\SetKwFunction{notifyCancelled}{updateCancel}
	\SetKwFunction{symmetryController}{symmetryController}

	\DontPrintSemicolon
	
	\Fn{\CDCL{$\varphi$: CNF formula, {\color{red}\C: symmetry controller}}
		\textbf{returns} $\true$ if $\varphi$ is \sat and $\false$ otherwise}
	{
		$dl \gets 0$ \tcp*{Current decision level}
		$\alpha \gets \emptyset$\;
		\While {not all variables are assigned} {
			$isConflict \gets$ \unitPropagation{}\;
			{
				\color{red} 
				\C.\notifyAssigned{$\alpha$}\;\label{algo:cdcl_sym:notify}
				$isReduced \gets$ \C.\isNotMinimal{$\alpha$}\;\label{algo:cdcl_sym:not_minimal}
			}
			
			\If{$isConflict \, {\color{red}||\, isReduced}$} {
				\If{dl == 0}
				{
					\Return \false
					\tcp*{$\varphi$ is $\unsat$}
				}
				\If{\color{red}$isConflict$}
				{
					$\omega \gets$ \analyzeConflict{}\;
				}
				\Else {
					{
						\color{red}$\omega \gets$ \C.\SBP{$\alpha$}\;\label{algo:cdcl_sym:gen_esbp}
					}
				}
			$\varphi \gets \varphi \cup \{\omega$\} \;
				$dl \gets$ \backjumpPolicy{}\;
				{\color{red}\C.\notifyCancelled{$\alpha$}\;} \label{algo:cdcl_sym:cancel}
			}
			\Else{
				$\alpha \gets \alpha\, \cup $ \assignNewLiteral{}\; 
			$dl \gets dl+1$\;
			}
		}
		\Return \true
		\tcp*{$\varphi$ is $\sat$}
		
	}
	\caption{the CDCLSym SAT Solving Algorithm.}
	\label{algorithm:cdcl_cosy}
\end{algorithm}


 
 The symmetry controller is initially given a set of symmetries $G$ \footnote{The generators of the group of symmetries.}. It observes the behavior of the SAT engine and updates its internal data according to the current assignment, to keep track of the status of the symmetries. This observation is \emph{incremental}: whenever a literal is assigned or canceled, the symmetry controller updates the status of all the symmetries. This corresponds to \cref{algo:cdcl_sym:notify,algo:cdcl_sym:cancel} of Algorithm~\ref{algo:cdcl}. When the controller detects that the
 current assignment cannot be a \emph{lex-leader} (\cref{algo:cdcl_sym:not_minimal}), it generates the
 corresponding \emph{esbp} (\cref{algo:cdcl_sym:gen_esbp}).
 
 \medskip\noindent In the remainder of this section, functions
 composing the symmetry controller are detailed.
 
 \subsubsection{Symmetries Status Tracking.} The \texttt{updateAssign},
 \texttt{updateCancel} and \texttt{isNot\-LexLeader} functions (\Cref{algo:keep_status}) track the status of symmetries based on
 Proposition~\ref{prop:status} ; there, resides th  core of our algorithm.
 
 All these functions rely on the $\track$ structure: a map of variables indexed
 by permutations. Initially, $\track[g] = \min_{\prec}(\support_g)$ for all $g \in G$ according to the ordering relation and all
 permutations are marked \textit{active}. 
 
 For each permutation, $g$, the symmetry controller keeps track of the smallest
 variable $\track[g]$ in the support of $g$ such that $\track[g]$ and
 $g^{-1}(\track[g])$ does not have the same value in the current assignment. If
 one of the two variables is not assigned, they are considered  to have different values.
 
 When new literals are assigned, only active symmetries need to have their
 $\track[g]$ updated (\cref{alg:esbp:update}). This update is done thanks to a while
 loop (\cref{alg:esbp:loop1,alg:esbp:loop2}).
 
 When literals are canceled, we need to update the status of symmetries for
 which some variable $v$ before $\track[g]$, or $g^{-1}(v)$, becomes unassigned
 (\cref{alg:esbp:min}). Symmetries that were inactive may be reactivated (\cref{alg:esbp:reactivate}).
 
 The current assignment is not a \textit{lex-leader} if some symmetry $g$ is a
 reducer. This is detected by comparing the value of $\track[g]$ with the value
 of $g^{-1}(\track[g])$ (\cref{alg:esbp:notmin}). The function \texttt{isNotLexLeader} also
 marks symmetries as \emph{inactive} when appropriate (\cref{alg:esbp:inactive1,alg:esbp:inactive2}).
 
 \subsubsection{Generation of the \emph{esbp}.} When the current assignment
 cannot be a \textit{lex-leader}, some symmetry $g$ is a reducer. The function
 \texttt{generateEsbp} computes the $\eta(\alpha, g)$ of Definition~\ref{def:eta},
the effective symmetry-breaking predicate of Proposition~\ref{prop:eta}. This will
 prevent the CDCL engine to explore further the current partial assignment.
 
 \begin{algorithm}[!htbp]
 \SetKwProg{Fn}{function}{}{}
  \SetKwFunction{notifyAssigned}{updateAssign}
  
  \Fn{\notifyAssigned{$\alpha$: assignment}
  }{
   \ForEach{active $g \in  G$}{\label{alg:esbp:update}
    $v \gets \track[g]$\;
    \While{ $\{v, g^{-1}(v)\}\subseteq\alpha $ \textbf{or} $\{\neg v, \neg g^{-1}(v)\} \subseteq\alpha$ }{ \label{alg:esbp:loop1}
     $v \gets$ next variable in $\Vars_g$\; \label{alg:esbp:loop2}
    }
    $\track[g] \gets v$
   }
  }
  
  \Fn{\notifyCancelled{$\alpha$: assignment}}{
   \ForEach{$g \in G$}{
    $u \gets \min \{ v \in \Vars_g \mid \{v, \neg v\} \cap \alpha = \emptyset \text{ or } \{g^{-1}(v), \neg g^{-1}(v)\} \cap \alpha = \emptyset \}$\;\label{alg:esbp:min}
    \If{$u \preceq \track[g]$}{
     mark $g$ as active\;\label{alg:esbp:reactivate}
     $\track[g] \gets u$\;
    }
    
   }
  }
  
  \Fn{\isNotMinimal{$\alpha$: assignment}}{
   \ForEach{active $g \in G$}{
    $v \gets \track[g]$\;
    \If{$\{v, \neg g^{-1}(v)\}\subseteq\alpha$} {\label{alg:esbp:notmin}
     \Return \true  \tcp*{$g$ is a reducer}
    }
    \If{$\{\neg v, g^{-1}(v)\}\subseteq\alpha$}{\label{alg:esbp:inactive1}
     mark $g$ as inactive \label{alg:esbp:inactive2}
     \tcp*{$g$ can't reduce $\alpha$ or its extensions}
    }
    
   }
   \Return \false
  }
  
  
  \Fn{\SBP{$\alpha$: assignment} \textbf{returns} $\omega$: generated $esbp$}
  {
   $\omega \gets \{\}$\;
   $g \gets$ the reducer of $\alpha$ detected in \isNotMinimal\;
   $v \gets min(\Vars_g)$\;
   $u \gets \track[g] $\;
   \While{$u \neq v$}{
    \leIf{$v \in \alpha$} {$\omega \gets \omega \cup \{\neg v\}$}{$\omega \gets \omega \cup \{v\}$}
    \leIf{$g^{-1}(v) \in \alpha$} {$\omega \gets \omega \cup \{\neg g^{-1}(v)\}$}{$\omega \gets \omega \cup \{g^{-1}(v)\}$}
    $v \gets$ next variable in $\Vars_g$
   }
   $\omega \gets \omega \cup \{\neg v, g^{-1}(v)\}$\;
   \Return $\omega$
  }
  \caption{the functions keeping track of the status of the symmetries and generating the \emph{esbp}.}
  \label{algo:keep_status}
  
 \end{algorithm}%
 
 
 
 \subsection{Illustrative example}
  Let us illustrate the previous concepts and algorithms on a simple example. Let the ordering relation $x_1 \prec x_2 \prec x_3 \prec x_4
 \prec x_5 \prec x_6\ \mid \false < \true$, and two generators:\\
 $G = \{
  g_1 = (x_1 \enskip x_2)(x_4 \enskip x_5),g_2 = (x_1 \enskip x_4)(x_2 \enskip x_5)(x_3 \enskip x_6)\}$
 (written in cycle notation with opposite cycles omitted). Their
 respective supports sorted according to ordering relation are, $\support_{g_{2}} = \{x_1, x_2, x_4, x_5\}$ and
 $\support_{g_2} = \{x_1, x_2, x_3, x_4, x_5, x_6\}$.
 
 On the assignment $\alpha = \emptyset$, both permutations are active and
 $\track[g_1] = \track[g_1] = x_1$. When the solver updates the assignment to $\alpha = \{ x_4\}$, both permutations remain active and $\track[g_1] = \track[g_2] = x_1$. On the assignment $\alpha = \{  x_4,  x_1\}$, the symmetry controller updates $\track[g_2]$ to $x_2$, while $\track[g_1]$ remains unchanged. On the assignment $\alpha = \{  x_4,  x_1, \neg x_2 \}$, $g_1.\alpha = \{  x_5,  x_2, \neg x_1
 \}$, which is smaller than $\alpha$ (because $ x_1 \in \alpha$ and $ \neg x_1 \in g.\alpha$):
 $g_1$ is a reducer of $\alpha$. The symmetry controller then generates the
 corresponding \textit{esbp} $\omega = \{ \neg x_1, x_2 \}$.
 
 
\section{Implementation and Evaluation}\label{sec:eval}
In this section, we first highlight some details on our implementation of the
symmetry controller. Then, we experimentally assess the performance of our
algorithm against three other state-of-the-art tools.
\subsection{\libdsb{}: an efficient implementation of the symmetry controller}
We have implemented our method in a C++ library called \libdsb{} (1630 LoC). It
implements a symmetry controller as described in the previous section, and can
be interfaced with virtually any CDCL SAT solver. \libdsb{} is released under
GPL~v3 license and is available at \url{https://github.com/lip6/cosy}.
\subsubsection{Heuristics and Options.} Let us recall that finding the optimal
ordering of variables (with respect to the exploitation of symmetries) is
NP-hard~\cite{luks.04.amai}, so the choice for this ordering is heuristic.
\libdsb{} offers several possibilities to define this ordering:
\begin{itemize}
 
 \item a naive ordering, where variables are ordered by the lexicographic
 order of their names;
 
 \item an ordering based on occurrences, where variables are sorted
 according to the number of times they occur in the input formula. The
 lexicographic order of variable names is used for those having the same number of
 occurrences;
 
 \item an ordering based on symmetries, where variables belonging to the
 same orbit (under the given set of symmetries) are grouped together. Orbit are
 ordered by their numbers of occurrences.
 
\end{itemize}
The ordering of assignments we use in this paper orders positive literals
before negative ones (thus, $\true < \false$), but using the converse
ordering does not change the overall method. However, it can impact the
performance of the solver on some instances, so that it is an option of the
library.
All the symmetries we used for the presentation of our approach are
permutations of variables. Our method straightforwardly extends to permutations of literals, also known as \emph{value permutations} \cite{biere2009handbook}.

\subsubsection{Integration in \minisat{}.} We show how to integrate \libdsb{}
to an existing solver, through an example of \minisat{}~\cite{een2003extensible}.
First, we need an adapter that allows the communication between the solver and
\libdsb{} (30 LoC). Then, we adapt Algorithm \ref{algo:cdcl} to the different
methods and functions of \minisat{}. In particular, the function
\texttt{updateAssign} is moved into the \texttt{uncheckEnqueue} function of
\minisat{} (2 LoC). The \texttt{updateCancel} function is moved to the
\texttt{cancelUntil} function of \minisat{} that performs the backjumps (2
LoC). The \texttt{isNotLexLeader} and \texttt{generateEsbp} functions are
integrated in the \texttt{propagate} function of \minisat{} (30 LoC). This is
to keep track of the assignments as soon as they occur, then the
\textit{esbp} is produced as soon as an assignment is identified as not being
\emph{lex-leader}. Initialization issues are located in the main function of
\minisat (15 LoC).
The integration of \libdsb{} increases \minisat{} code by~3\%.
 
\subsection{Evaluation}
% \textcolor{red}{revoir le commentaire lorsqu'on aura tous les tableaux...}
%
% \textcolor{red}{rajouter un paragraphe sur une expérience pour lever le doute
% sur le fait que nos perfs soient liees a un effet de bord d'implementation.
% Essayer de lancer notre outil avec l'ordre d'un autre outil. On ferait une
% analyse en quelques lignes en indiquant la variablite observee. Experience =
% ordre true/false + l'ordre de breakID.}
This section presents the evaluation of our approach. All experiments have been
performed with our modified \minisat{} called \cdclsym{}. The symmetries of the
SAT problem instances have been computed by two different state-of-the-art
tools \saucy{}~\cite{katebi2010symmetry} and
\bliss{}~\cite{JunttilaKaski:ALENEX2007}. For a given group of symmetries, the
first tool generates less permutations to represent the group than the second
one, but it is slower than the other one.
We selected symmetric instance  of the SAT contests\cite{jarvisalo2012international}
from 2012 to 2017, we call a symmetric instance a CNF instances for which \bliss{} finds
%at least 2\% of the variables involved in some
 symmetries that could be computed in at most $1000$ seconds of CPU time. We obtained a total of $1350$
symmetric instances (discarding repetitions) out of $3700$ instances in total.
All experiments have been conducted using the following conditions: each solver
has been run once on each instance, with a time-out of 5000 seconds (including
the execution time of the symmetries generation except for \minisat) and limited
to 8 GB of memory. Experiments were executed on a computer with an Intel Xeon
X7460 2.66 GHz featuring 24 cores and 128 GB of memory, running a Linux 4.4.13,
along with g++ compiler version 6.3.
We compare \cdclsym{} using the occurrence order, value symmetries, and without
\emph{lex-leader} forcing, against:
\begin{itemize}
 
 \item \minisat{}, as the reference solver without symmetry handling
 \cite{een2003extensible};
 
 \item \shatter{}, a symmetry breaking preprocessor described in \cite{aloul06},
 coupled with the \minisat{} SAT engine;
 
 \item \breakid{}, another symmetry breaking preprocessor, described in
 \cite{devriendt2016improved}, also coupled with the \minisat{} SAT engine.
 
\end{itemize}
Each $\sat$ solution was successfully checked against the initial CNF. For \unsat
situations, there is no way to provide an $\unsat\,$ certificate in presence of
symmetries. Nevertheless, we checked our results were also computed by the
other measured tools. Unfortunately, out of the 1350 benchmarked formulas, we
have no proof or evidence for the 15 $\unsat$ formulas computed by $\cdclsym{}$
only.
Results are presented Tables in~\ref{table:benchUNSAT}, \ref{table:benchSAT},
and \ref{tab:par2}. We report the number of instances solved within the time
and memory limits for each solver and category. We separate the UNSAT instances
(\Cref{table:benchUNSAT}) from the SAT ones (\Cref{table:benchSAT}). Besides
the reference with no symmetry (column \minisat{}), we have compared the
performance of the three tools when using symmetries computed by \saucy{} (see
Table~\ref{table:unsat:saucy} and Table~\ref{table:sat:saucy}), and \bliss{}
(see Table~\ref{table:unsat:bliss} and Table~\ref{table:sat:bliss}). Rows
correspond to groups of instances: from each edition of the SAT contest, and
when possible, we separated applicative instances (app$\langle x \rangle$ where
$\langle x \rangle$ indicates the year) from hard combinatorial ones
(hard$\langle x \rangle$). This separation was not possible for the editions
2015 and 2017 (all2015 and all2017). The total number of instances for each
bench is indicated between parentheses. For each row, the cells corresponding
to the tools solving the most instances (within time and memory limits) are
typeset in bold and grayed out. Table~\ref{tab:par2} shows the cumulative and
average PAR-2 times of the evaluated tools. PAR-2 measure is used in SAT competition,
it corresponds to the sum of cumulative time of solved instances with 2 times t
he timeout of unsolved instances.
 
\begin{table}[t]
 \resizebox{1 \textwidth}{!}{
  \subfloat[With \saucy]{%
   \begin{tabular}{l|ccccc}
    Benchmark  &\texttt{MiniSAT} & \texttt{Shatter} & \texttt{BreakID} & \texttt{MiniSym}\\
    \midrule
    app2016 (134) & 18 & 19 & \cellcolor{gray!30}\textbf{20} & 17\\
    app2014 (161) & 23 & 23 & 22 & \cellcolor{gray!30}\textbf{24}\\
    app2013 (145) & 6 & 8 & 8 & \cellcolor{gray!30}\textbf{10}\\
    app2012 (367) & 115 & 115 & \cellcolor{gray!30}\textbf{120} & \cellcolor{gray!30}\textbf{120}\\
    \hline
    hard2016 (128) & 8 & 17 & \cellcolor{gray!30}\textbf{50} & 42\\
    hard2014 (107) & 9 & 24 & \cellcolor{gray!30}\textbf{30} & 29\\
    hard2013 (121) & 12 & 24 & \cellcolor{gray!30}\textbf{48} & 29\\
    hard2012 (289) & 86 & 84 & 88 & \cellcolor{gray!30}\textbf{93}\\
    \hline
    all2017 (124) & 8 & 14 & \cellcolor{gray!30}\textbf{15} & 14\\
    all2015 (65) & 9 & 8 & 8 & \cellcolor{gray!30}\textbf{10}\\
    \hline
    TOTAL (no dup)  & 261 & 302 & \cellcolor{gray!30}\textbf{371} & 345\\
    \label{table:unsat:saucy}
   \end{tabular}
  }
  \hspace{1em}
  \subfloat[With \bliss]{%
   \begin{tabular}{l|ccccc}
    Benchmark  &\texttt{MiniSAT} & \texttt{Shatter} & \texttt{BreakID} & \texttt{MiniSym}\\
    \midrule
    app2016 (134) & 18 & \cellcolor{gray!30}\textbf{21} & 18 & 19\\
    app2014 (161) & 23 & 21 & 20 & \cellcolor{gray!30}\textbf{24}\\
    app2013 (145) & 6 & 7 & 10 & \cellcolor{gray!30}\textbf{11}\\
    app2012 (367) & 115 & 106 & 114 & \cellcolor{gray!30}\textbf{123}\\
    \hline
    hard2016 (128) & 8 & 11 & \cellcolor{gray!30}\textbf{79} & 77\\
    hard2014 (107) & 9 & 45 & 40 & \cellcolor{gray!30}\textbf{53}\\
    hard2013 (121) & 12 & 51 & \cellcolor{gray!30}\textbf{56} & 54\\
    hard2012 (289) & 86 & 69 & 90 & \cellcolor{gray!30}\textbf{93}\\
    \hline
    all2017 (124) & 8 & 14 & \cellcolor{gray!30}\textbf{15} & \cellcolor{gray!30}\textbf{15}\\
    all2015 (65) & \cellcolor{gray!30}\textbf{9} & 7 & 8 & 8\\
    \hline
    TOTAL (no dup) & 261 & 324 & 415 & \cellcolor{gray!30}\textbf{439}\\
    \label{table:unsat:bliss}
   \end{tabular}
  }
 }
 \vspace*{0.1cm}
 \caption{Comparison of different approaches on the $\unsat$ instances of the benchmarks of the six last editions of the SAT competition.}
 \label{table:benchUNSAT}
\end{table}
\begin{table}
 \resizebox{1 \textwidth}{!}{
  \subfloat[With \saucy]{%
   \begin{tabular}{l|ccccc}
    Benchmark  &\texttt{MiniSAT} & \texttt{Shatter} & \texttt{BreakID} & \texttt{MiniSym}\\
    \midrule
    app2016 (134) & 20 & \cellcolor{gray!30}\textbf{22} & 21 & 20\\
    app2014 (161) & \cellcolor{gray!30}\textbf{24} & \cellcolor{gray!30}\textbf{24} & \cellcolor{gray!30}\textbf{24} & 22\\
    app2013 (145) & 34 & 35 & 35 & \cellcolor{gray!30}\textbf{43}\\
    app2012 (367) & 121 & 112 & 119 & \cellcolor{gray!30}\textbf{126}\\
    \hline
    hard2016 (128) & \cellcolor{gray!30}\textbf{0} & \cellcolor{gray!30}\textbf{0} & \cellcolor{gray!30}\textbf{0} & \cellcolor{gray!30}\textbf{0}\\
    hard2014 (107) & 14 & \cellcolor{gray!30}\textbf{17} & \cellcolor{gray!30}\textbf{17} & 14\\
    hard2013 (121) & 23 & 23 & \cellcolor{gray!30}\textbf{24} & 22\\
    hard2012 (289) & 135 & 141 & \cellcolor{gray!30}\textbf{143} & 138\\
    \hline
    all2017 (124) & 23 & 20 & 26 & \cellcolor{gray!30}\textbf{27}\\
    all2015 (65) & \cellcolor{gray!30}\textbf{7} & 5 & \cellcolor{gray!30}\textbf{7} & 6\\
    \hline
    TOTAL (no dup)  & 325 & 323 & \cellcolor{gray!30}\textbf{337} & 335\\
    \label{table:sat:saucy}
    
   \end{tabular}
  }
  \hspace{1em}
  \subfloat[With \bliss]{%
   \begin{tabular}{l|ccccc}
    Benchmark  &\texttt{MiniSAT} & \texttt{Shatter} & \texttt{BreakID} & \texttt{MiniSym}\\
    \hline
    app2016 (134) & 20 & 20 & \cellcolor{gray!30}\textbf{22} & 20\\
    app2014 (161) & \cellcolor{gray!30}\textbf{24} & \cellcolor{gray!30}\textbf{24} & 23 & 22\\
    app2013 (145) & \cellcolor{gray!30}\textbf{34} & 32 & 30 & 33\\
    app2012 (367) & \cellcolor{gray!30}\textbf{121} & 112 & 120 & 118\\
    \hline
    
    hard2016 (128) & \cellcolor{gray!30}\textbf{0} & \cellcolor{gray!30}\textbf{0} & \cellcolor{gray!30}\textbf{0} & \cellcolor{gray!30}\textbf{0}\\
    hard2014 (107) & 14 & 14 & 17 & \cellcolor{gray!30}\textbf{18}\\
    hard2013 (121) & 23 & 24 & \cellcolor{gray!30}\textbf{26} & 25\\
    hard2012 (289) & 135 & 134 & 141 & \cellcolor{gray!30}\textbf{142}\\
    \hline
    all2017 (124) & 23 & 25 & 26 & \cellcolor{gray!30}\textbf{29}\\
    all2015 (65) & \cellcolor{gray!30}\textbf{7} & 5 & 6 & 6\\
    \hline
    TOTAL (no dup) & 325 & 316 & 334 & \cellcolor{gray!30}\textbf{336}\\
    \label{table:sat:bliss}
   \end{tabular}
  }
 }
 \vspace*{0.1cm}
 \caption{Comparison of different approaches on the $\sat$ instances of the benchmarks of the six last editions of the SAT competition.}
 \label{table:benchSAT}
\end{table}
\begin{table}[h!]
 \centering\footnotesize
 \subfloat[With \saucy]{%
  \begin{tabular}{l|c|c}
   Solver & PAR-2 sum & PAR-2 avg \\
   \hline
   \texttt{MiniSAT} & 8\,074\,348       & 5\,981        \\
   \texttt{Shatter} & 7\,770\,434        & 5\,756        \\
   \cellcolor{gray!30}\texttt{BreakID} & \cellcolor{gray!30}6\,909\,999      & \cellcolor{gray!30}5\,119 \\
   \texttt{MiniSym} & 7\,229\,700           & \ 5\,355       
   \label{par2-saucy}
  \end{tabular}
 }
 \hspace{1em}
 \subfloat[With \bliss]{%
  \begin{tabular}{l|c|c}
   Solver & PAR-2 sum & PAR-2 avg \\
   \hline
   \texttt{MiniSAT} & 8\,074\,348        & 5\,981        \\
   \texttt{Shatter} & 7\,517\,556        & 5\,569        \\
   \texttt{BreakID} & 6\,444\,954        & 4\,774        \\ 
   \cellcolor{gray!30}\texttt{MiniSym} & \cellcolor{gray!30}6\,245\,448        & \cellcolor{gray!30}\ 4\,626       
   \label{par2-bliss}
  \end{tabular}
 }
 \vspace*{0.1cm}
 \caption{Comparison of PAR-2 times (in seconds) of the benchmarks on the six last editions of the SAT competition.}
 \label{tab:par2}
\end{table}
We observe that \cdclsym{} with \saucy{} solves the most instances in only half
of the $\unsat$ categories. However, with $\bliss{}$, $\cdclsym{}$ solves the most
instances in all but four of the $\unsat$ categories; it then also solves the
highest number of instances among its competitors. This shows the interest
of our approach for $\unsat$ instances. Since symmetries are used to reduce the
search space, we were expecting that it will bring the most performance gain
for $\unsat$ instances.
The situation for $\sat$ instances is more mitigated (Table~\ref{table:benchSAT}),
especially when using \saucy{}. Again, this is not very surprising: our method
may cut the exploration of a satisfying assignment because it is not a
\textit{lex-leader}. This delays the discovery of a satisfying assignment. The
other tools suffer less from such a delay, because they rely on symmetry
breaking predicates generated in a pre-processing step. Also, when seeing the
global results of \minisat{}, we can globally state that the use of symmetries
in the case of satisfiable instances only offers a marginal improvement.
\begin{figure}[!htbp]
 \centering
 \subfloat[with \saucy]{{\includegraphics[scale=0.36]{img/saucy-result}}}%
 \qquad
 \subfloat[with \bliss]{{\includegraphics[scale=0.36]{img/bliss-result}}}%
 \caption{cactus plot  total number of instances}%
 \label{fig:cactus}%
\end{figure}
We observe that performances our tool are better with \bliss{} than with
\saucy{} (see fig~\ref{fig:cactus}). We explain it as follows: \saucy{} is
known to compute fewer generators for the group of symmetries than \bliss{}.
Since the larger the symmetries set is, the earlier the detection of an
\emph{evidence} that an assignment is not a \textit{lex-leader} will be, we
generate less symmetry-breaking predicates (only the effective ones). This is
shown in Table~\ref{tab:sbp}; \cdclsym{} generates an order of magnitude fewer
predicates than \breakid{}.
\begin{table}[!htbp]
 \resizebox{1 \textwidth}{!}{
  \subfloat[With \saucy]{%
   \begin{tabular}{l|c|c}
    Number of SBPs & \texttt{BreakID}  & \texttt{MiniSym}\\
    \hline
    $\unsat$ (316) & 12\,088\,433  & \cellcolor{gray!30}1\,579\,623 \\
    $\sat$  (312)    & 13\,839\,689  & \  \cellcolor{gray!30}359\,352     
    \label{sbp-saucy}
   \end{tabular}
  }
  \hspace{1em}
  \subfloat[With \bliss]{%
   \begin{tabular}{l|c|c}
    Number of SBPs & \texttt{BreakID}  & \texttt{MiniSym}\\
    \hline
    $\unsat$ (399) &  2\,576\,349 &  \cellcolor{gray!30}913\,339\\
    $\sat$  (320) & 12\,179\,513  & \  \cellcolor{gray!30}457\,452 
    \label{sbp-bliss}
   \end{tabular}
  }
 }
 \vspace*{0.1cm}
 \caption{Comparison of the number of generated SBPs each time \breakid{} and 
  \cdclsym{} both compute a verdict (number of verdicts between parentheses).}
 \label{tab:sbp}
\end{table}
We also conducted experiments on highly symmetrical instances (all variables
are involved in symmetries), whose results are presented
in~\Cref{table:bench2}. The performance of \breakid{} on this benchmark is
explained by a specific optimization for the \emph{total symmetry groups} that
are found in these examples, that is neither implemented in \shatter{} nor in
\cdclsym{}. However, the difference between \breakid{} and \cdclsym{} is rather
thin when using $\bliss{}$. Our tool still outperforms \shatter{} on this
benchmark.
\begin{table}[!htbp]
 \resizebox{1 \textwidth}{!}{
  \subfloat[With \saucy]{%
   \begin{tabular}{l|cccc}
    Benchmark  & \minisat{} & \shatter{} & \breakid{} & \cdclsym{} \\
    \hline
    battleship(6) & \cellcolor{gray!30}\textbf{5} &\cellcolor{gray!30}\textbf{5} & \cellcolor{gray!30}\textbf{5} & \cellcolor{gray!30}\textbf{5}\\
    chnl(6) &  4 & \cellcolor{gray!30}\textbf{6} & \cellcolor{gray!30}\textbf{6} & \cellcolor{gray!30}\textbf{6}\\
    clqcolor(10)  &  3 & 4 & 5 & \cellcolor{gray!30}\textbf{6} \\
    fpga(10) &   6 & \cellcolor{gray!30}\textbf{10} & \cellcolor{gray!30}\textbf{10} & \cellcolor{gray!30}\textbf{10}\\
    hole(24) &   10 & 12 & \cellcolor{gray!30}\textbf{23} & 11\\
    hole shuffle(12)  &  1  & 2 & \cellcolor{gray!30}\textbf{12} & 3\\
    urq(6) &   1  & 2 & \cellcolor{gray!30}\textbf{6} & 2\\
    xorchain(2)  &  1 & 1 & \cellcolor{gray!30}\textbf{2} & \cellcolor{gray!30}\textbf{2} \\
    \hline
    TOTAL & 31 & 42 & \cellcolor{gray!30}\textbf{69} & 45\\
   \end{tabular}
  }
  \hspace{1em}
  \subfloat[With \bliss]{%
   \begin{tabular}{l|cccc}
    Benchmark  & \minisat{} & \shatter{} & \breakid{} & \cdclsym{} \\
    \hline
    battleship(6) & 5 & 5 & 5 & \cellcolor{gray!30}\textbf{6} \\
    chnl(6) &  4 &  \cellcolor{gray!30}\textbf{6} & \cellcolor{gray!30}\textbf{6} & \cellcolor{gray!30}\textbf{6} \\
    clqcolor(10)  &  3 & 5 & 8 & \cellcolor{gray!30}\textbf{10} \\
    fpga(10) &   6 & \cellcolor{gray!30}\textbf{10} & \cellcolor{gray!30}\textbf{10} & \cellcolor{gray!30}\textbf{10} \\
    hole(24) &   10 &\cellcolor{gray!30}\textbf{24} & \cellcolor{gray!30}\textbf{24} & 23 \\
    hole shuffle(12)  &  1  &  3 & \cellcolor{gray!30}\textbf{7} & 4\\
    urq(6) &   1  & 2 & \cellcolor{gray!30}\textbf{6} & 5\\
    xorchain(2)  &  1 & 1 & \cellcolor{gray!30}\textbf{2} & \cellcolor{gray!30}\textbf{2}\\
    \hline
    TOTAL & 31 & 56 & \cellcolor{gray!30}\textbf{68} & 66 \\
   \end{tabular}
  }
 }
 \vspace*{0.1cm}
 \caption{Comparison of the tools on 76 highly symmetric $\unsat$ problems.}
 \label{table:bench2}
\end{table}
\section{Some optimization}
Usage of symmetry property dynamically allows the solver to adapt classical heuristics and symmetry based one on the fly.
For example, some restart heuristics are based on the number of conflicts, taking into account injection of esbp may impact
the performance of the overall SAT solver. 
\subsection{Adapt heuristics dynamically}
Other heuristics on the symmetry handling may increase the performance. We present here some of them.
In some cases, multiple permutations can be reducers at the same time, and each one generates different symmetry breaking constraints.
The backtrack and the pruning capacity depends heavily on the chosen constraint. In our library, the first reducer permutation generates the \textit{esbp}. 
Another point concerns the injection of the symmetry breaking predicates. Two choices are possibles, first,
before the unit propagation and second, at the end of the propagation. This choice will impact the solver behavior.
In the first case, esbp takes the lead over the classical conflict (if its occurs). Conversely, in the second case, the classical conflict
takes the lead over esbp. This can be, especially, useful on SAT problems because esbp can eliminate non lex-leader SAT assignment.
To emphasize this behavior, the conflict of the esbp can be ignored in the sense that the conflict clause is just added into the clause database and so will participate to the next unit propagation. This gives to the solver the ability to find a solution symmetrical branch but avoid to get multiple times on non-minimal part of search. I can be useful if we know in advance that the problem is satisfiable.

\subsection{Change the Order Dynamically}
As seen before, the ordering relationship between variables influence the minimal value of each orbit (lex-leader) and the generated constraints. The symmetry controller is "waiting" for the solver that assigns the variables that allows it to decide if the current assignment is the lex-leader.
The main idea to change dynamically the order, and so the lex-leader, is that the symmetry breaking order follows the decision heuristics of the solver, then symmetry controller can decide quickly if the current assignment is minimal.
Changing this order dynamically is possible with some requirements: all esbps  and all deduced 
clauses from a symmetry breaking predicates need to be removed. If these constraints are not deleted,
the correctness of the algorithm is not guaranteed.
% inconsistencies may appear and a satisfiable problem become unsatisfiable. 
%So, changing this order will force the solver to forget all learned clauses that makes the strength of CDCL solver. 

\subsection{Impact of the sign in variable ordering}
With the same variables ordering, swapping the value thus, $\true < \false$ or $\false < \true$
may impact drastically the performance of the solver.
To illustrate it, consider the pigeonhole problem with 100 holes and 101 pigeons with 
the increasing order of the variables and change only the sign.
With  $\false < \true$, the solver generates 20\,619 esbps, and takes 13.8 seconds to solve it.
With the reverse order ($\true < \false$), it generates 33\,263 esbps and solves it in 93.4 seconds.
\Cref{table:compare_sign_order} shows this difference on 500 symmetric instances with
a scatter plot that compares the same variable order with $\true < \false$ and $\false < \true$,
 \texttt{MiniSymFT} is the solver in which $\false < \true$, and  \texttt{MiniSymTF} is the solver in which $\true < \false$.
On the left, we compare the computation time of the solver. 
As we can observe \texttt{MiniSymTF} is more efficient on some $\unsat$ instances (red points in the figure).
The right figure shows the number of generated esbp by solvers in a log scale. On the large majority of instances, they generate 
approximately the same number of esbps. But the difference can an order of magnitude higher. This can be due to the time of
execution of the solver and/or the impact of the sign of the constraints.

\begin{figure}[!htbp]
 \subfloat[Time]{%
  \includegraphics[width=.3\textheight]{img/scatter_compare}%
 }
 \hspace{1em}
 \subfloat[Number of esbp]{%
  \includegraphics[width=.375\textheight]{img/scatter_esbp_compare}%
 }
 \vspace*{0.1cm}
 \caption{Comparison of the order with different signs on 500 symmetric instances.}
 \label{table:compare_sign_order}
\end{figure}

\texttt{MiniSymTF} is generally better and is the default choice in the library.
If it is running on a specific application, reverse order can be chosen if it performs better results.

\section{Conclusion}
SymmSAT uses same the principles as static symmetry breaking approaches but operates dynamically by 
injecting \textit{effective symmetry breaking} during the search.
This overcomes, the main problem of the static approaches, that they
generate many \textit{sbps} that are not effective in the solving (size of the
generated formulas, overburden of the unit propagation procedure, etc.).
The idea we brought it is to break symmetries \emph{on the fly}: when the current
partial assignment cannot be a prefix of a \textit{lex-leader} (of an orbit),
an \textit{esbp} that prunes this forbidden assignment and all its extensions are generated. 
This approach is implemented in the C++ library called \libdsb{}. It is an
off-the-shelf component that can be interfaced with virtually any CDCL SAT
solver. \libdsb{} is released under GPL license and is available at
\url{https://github.com/lip6/cosy}.
 
The extensive evaluation of our approach on the symmetric formulas of the 
 SAT contests from 2012 to 2017 shows that it outperforms the state-of-the-art techniques, in
particular on unsatisfiable instances, which are the hardest class of the
problem.