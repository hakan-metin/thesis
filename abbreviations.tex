%%% Command abbreviations =====================================================
\newcommand{\cc}[1]{\multicolumn{1}{|c|}{#1}}
\newcommand{\ccr}[1]{\multicolumn{1}{c|}{#1}}
\newcommand{\xll}[1]{\multirow{2}{*}{#1}}
\newcommand{\xlll}[1]{\multirow{3}{*}{#1}}

\newcommand{\result}[1]{#1}


%\newcounter{theo}[section]\setcounter{theo}{0}
%\renewcommand{\thetheo}{\arabic{section}.\arabic{theo}}
%\newenvironment{theo}[2][]{%
%	\refstepcounter{theo}
%	
%	% Code for box design goes here.
%	
%	\begin{mdframed}[]\relax}{%
%	\end{mdframed}

%\ifstrempty{#1}%
%% if condition (without title)
%{\mdfsetup{%
%		frametitle={%
%			\tikz[baseline=(current bounding box.east),outer sep=0pt]
%			\node[anchor=east,rectangle,fill=blue!20]
%			{\strut Theorem~\thetheo};}
%	}%
%	% else condition (with title)
%}{\mdfsetup{%
%		frametitle={%
%			\tikz[baseline=(current bounding box.east),outer sep=0pt]
%			\node[anchor=east,rectangle,fill=blue!20]
%			{\strut Theorem~\thetheo:~#1};}%
%	}%
%}%
%% Both conditions
%\mdfsetup{%
%	innertopmargin=10pt,linecolor=blue!20,%
%	linewidth=2pt,topline=true,%
%	frametitleaboveskip=\dimexpr-\ht\strutbox\relax%
%}
%\newcounter{definition}[section]\setcounter{theo}{0}
%\renewcommand{\thetheo}{\arabic{section}.\arabic{theo}}


%\usepackage{DejaVuSans}
%\usepackage[T1]{fontenc}


\newcounter{theo}[chapter]\setcounter{theo}{0}
\renewcommand{\thetheo}{\arabic{chapter}.\arabic{theo}}
\newenvironment{definition}[1][]{%
	\refstepcounter{theo}%
	\ifstrempty{#1}%
	{\mdfsetup{%
			frametitle={%
				\tikz[baseline=(current bounding box.east),outer sep=0pt]
				\node[anchor=east,rectangle,fill=blue!20]
				{\strut Definition~\thetheo};}}
	}%
	{\mdfsetup{%
			frametitle={%
				\tikz[baseline=(current bounding box.east),outer sep=0pt]
				\node[anchor=east,rectangle,fill=blue!20]
				{\strut Definition~\thetheo:~#1};}}%
	}%
	\mdfsetup{innertopmargin=10pt,linecolor=blue!20,%
		linewidth=2pt,topline=true,%
		frametitleaboveskip=\dimexpr-\ht\strutbox\relax
	}
	\begin{mdframed}%
	}{\end{mdframed}}


\newcounter{clem}[chapter]\setcounter{clem}{0}
\renewcommand{\theclem}{\arabic{chapter}.\arabic{clem}}
\newenvironment{proposition}[1][]{%
	\refstepcounter{clem}%
	\ifstrempty{#1}%
	{\mdfsetup{%
			frametitle={%
				\tikz[baseline=(current bounding box.east),outer sep=0pt]
				\node[anchor=east,rectangle,fill=green!20]
				{\strut Proposition~\theclem};}}
	}%
	{\mdfsetup{%
			frametitle={%
				\tikz[baseline=(current bounding box.east),outer sep=0pt]
				\node[anchor=east,rectangle,fill=green!20]
				{\strut Proposition~\theclem:~#1};}}%
	}%
	\mdfsetup{innertopmargin=10pt,linecolor=green!20,%
		linewidth=2pt,topline=true,%
		frametitleaboveskip=\dimexpr-\ht\strutbox\relax
	}
	\begin{mdframed}%
	}{\end{mdframed}}
	
	
	\newcounter{ctheo}[chapter]\setcounter{ctheo}{0}
	\renewcommand{\thectheo}{\arabic{chapter}.\arabic{ctheo}}
	\newenvironment{theorem}[1][]{%
		\refstepcounter{ctheo}%
		\ifstrempty{#1}%
		{\mdfsetup{%
				frametitle={%
					\tikz[baseline=(current bounding box.east),outer sep=0pt]
					\node[anchor=east,rectangle,fill=red!20]
					{\strut Theorem~\thectheo};}}
		}%
		{\mdfsetup{%
				frametitle={%
					\tikz[baseline=(current bounding box.east),outer sep=0pt]
					\node[anchor=east,rectangle,fill=red!20]
					{\strut Theorem~\thectheo:~#1};}}%
		}%
		\mdfsetup{innertopmargin=10pt,linecolor=red!20,%
			linewidth=2pt,topline=true,%
			frametitleaboveskip=\dimexpr-\ht\strutbox\relax
		}
		\begin{mdframed}%
		}{\end{mdframed}}
	
%\newenvironment{definition}[1]
%{\mdfsetup{
%		frametitle={\colorbox{white}{Definition: \space#1\space}},
%		innertopmargin=10pt,
%		frametitleaboveskip=-\ht\strutbox,
%		frametitlealignment=\center
%	}
%	\begin{mdframed}%
%	}
%	{\end{mdframed}}

%\newmdtheoremenv{definition}{Definition}{\bfseries}{\itshape}
%\newtcolorbox{definition}{title=Definition:}%
%{colframe=gray!50!black,fonttitle=\bfseries}{}

%\newtheorem{proposition}{Proposition}{\bfseries}{\itshape}
%\newtheorem{theorem}{Theorem}{\bfseries}{\itshape}

%%% Common abbreviations ======================================================
\usepackage{hyphenat}

\newcommand{\true}{\ensuremath{\top}}
\newcommand{\false}{\ensuremath{\bot}}

\newcommand{\sat}{\textsc{sat}}
\newcommand{\unsat}{\textsc{unsat}}


\newcommand{\Vars}{\ensuremath{\mathcal{V}}}
\newcommand{\Lits}{\ensuremath{\mathcal{L}}}

\newcommand{\Assignments}{\ensuremath{Ass}}


\newcommand{\Group}[0]{\ensuremath{\mathfrak{S}}}

\newcommand{\support}{\ensuremath{supp}}

\newcommand{\track}{\ensuremath{pt}}

% Automorphism tool 
\newcommand{\bliss}{\texttt{bliss}}
\newcommand{\saucy}{\texttt{saucy3}}
\newcommand{\nauty}{\texttt{nauty}}

\newcommand{\libdsb}{\textsf{cosy}}

% Solvers
\newcommand{\minisat}{\texttt{MiniSAT}}
\newcommand{\breakid}{\texttt{BreakID}}
\newcommand{\shatter}{\texttt{Shatter}}
\newcommand{\cdclsym}{\texttt{MiniSym}}

%\newcommand{\cdclsym}{\texttt{CDCLSym}}
\newcommand{\cdclsp}{\texttt{CDCLSp}}
\newcommand{\cdclsymsp}{\texttt{CDCLSymSp}}
\newcommand{\cdcl}{\texttt{CDCL}}



\newcommand{\colorsp}[1]{\color{blue}#1\color{black}}
\newcommand{\colorsym}[1]{\color{red}#1\color{black}}
\newcommand{\colormix}[1]{\color{blak}#1\color{black}}