\chapter{Symmetry in SAT}\label{chap:symmetryinsat}

This chapter presents the computation and usage of symmetry in SAT problems.

\section{Symmetry detection in SAT}

To find symmetries in SAT problem, the formula is transformed into colored graph
and an automophism tool is applied onto. Specifically, given a formula $\varphi$ with
$m$ clauses and $l$ literals over $n$ variables, the graph is constructed as follows:
\emph{clause nodes}: represent each of the $m$ clauses by a node with color 0;
\emph{literals nodes}: represent each of the $l$ literals by a node with color 1;
\emph{clauses edges}: connect each clause node to the node of the literals that appear in clause
\emph{boolean consistency edges}: connect each pair of literals that correspond to the same variables.

An automorphism tool detect symmetries in this graph, and outputs a set of permutations that keep the problem
invariant. This set is a permutation group and generators of the group that acts on the formula $\varphi$.

\hakan{optimisation du graph}
\hakan{example of graph}
Example:

\section{Symmetry breaking}

Permutations of a formula keeps the problem invariant. By definition, these permutations forms a group.
This group acts naturally on set of 


It follows an important property, for each assignment $\alpha$ that satisfies the formula $\varphi$ 

\subsection{Lexicographic leader}
\subsection{Compact CNF encoding of Lexicographic leader}

\subsection{Some special groups}




%% Local Variables:
%% TeX-master: "main.tex"
%% ispell-dictionary: "en_US"
%% mode: latex
%% mode: flyspell
%% coding: utf-8
%% End:
