\chapter{Symmetry and SAT}\label{chap:symmetryinsat}

%This chapter presents the computation and usage of symmetry in SAT problems.


The group of permutations of $\Vars$ (i.e. bijections from $\Vars$ to $\Vars$) is noted
$\Group(\Vars)$. The group $\Group(\Vars)$ naturally acts on the set of literals: for $g
\in \Group(\Vars)$ and a literal $\ell \in \Lits $, $g.\ell = g(\ell)$ if $\ell$ is a
positive literal, $g.\ell = \neg g(\neg \ell)$ if $\ell$ is a negative literal.
The group $\Group(\Vars)$ also acts on (partial) assignments of $\Vars$ as follows: for
$g \in \Group(\Vars)$, $\alpha \in \Assignments(\Vars)$, $g.\alpha = \{ g.\ell ~|~ \ell \in \alpha \}$. Let $\varphi$ be a formula, and $g \in \Group(\Vars)$. We say that $g\in \Group(\Vars)$ is a
symmetry of $ \varphi$ if for every \emph{complete} assignment $\alpha$, $\alpha
\models \varphi$ if and only if $g.\alpha \models \varphi$. The set of symmetries
of $\varphi$ is noted $S(\varphi) \subseteq \Group(\Vars)$.



The previous mathematical definitions of group theory is applied to the CNF formula.
So, the group of permutations of $\Vars$ (i.e. bijections from $\Vars$ to $\Vars$) is noted
$\Group(\Vars)$. We say that $g\in \Group(\Vars)$ is a symmetry of $ \varphi$ if following conditions holds:
\vspace{-1em}
\begin{itemize}[topsep=1em]
	\item permutation fixes the formula, $g(\varphi) =  \varphi$ 
	\item $g$  commutes with the negation: $g(\neg l)  = \neg g(l)$
\end{itemize}

The set of symmetries of $\varphi$ is noted $S(\varphi) \subseteq \Group(\Vars)$.
The sets of symmetries of a formula $\varphi$ preserves the satisfaction,
for every \emph{complete} assignment $\alpha$, 
$\alpha \models \varphi \leftrightarrow g(\alpha) \models \varphi$ for $g \in S(\varphi)$.
The group $S(\varphi)$ also acts on (partial) assignments of $\Vars$ as follows: for
$g \in S(\varphi)$, $\alpha \in \Assignments(\Vars)$, $g.\alpha = \{ g.\ell ~|~ \ell \in \alpha \}$.


The next section presents how to compute the set of \emph{generators} of a given formula.

%Let $G$ be a subgroup of $\Group(\Vars)$. The \emph{orbit of $\alpha$ under $G$} (or
%simply the \emph{orbit of $\alpha$} when $G$ is clear from the context) is the set
%$ [\alpha]_G=\{ g.\alpha \mid g \in G \}$. The lexicographic leader
%(\textit{lex-leader} for short) of an orbit $[\alpha]_G$ is defined by
%$min_<([\alpha]_G)$. This \textit{lex-leader} is unique because the lexicographic
%order is a total order.


\section{Symmetry detection in SAT}

For the detection of symmetries in SAT, we fist introduce the graph automorphism notion.
Given a colored graph $G = (V, E, \gamma)$, with vertex set $V \in  [1, n] $, edge set E and
$\gamma$ a function that apply a mapping : $V \rightarrow C$ where C is a set of \emph{colors}.
An automorphism of G is a permutation from its vertices $g :V \rightarrow V$ 
such that:
\begin{itemize}
	\item $\forall (u, v) \in E \implies (g(u), g(v)) \in E$
	\item $\forall v \in V, \gamma(v) = \gamma(g(v))$
\end{itemize}

The graph automorphism problem is to find if a given graph has a non trivial permutation group. 
The computational complexity of this algorithm is conjectured to be strictly between P and NP.
Several tools exists to tackle this problem like \saucy~\cite{katebi2010symmetry},
\bliss~\cite{JunttilaKaski:ALENEX2007}, \nauty~\cite{mckay2003nauty}, etc.


%\hakan{syntax semantic symmetries}\\
There exists different ways to encode a SAT problems, which leads to different symmetries in these problem.
When a symmetry depends on the structure of the problem, we say \emph{syntactic} symmetries. 
In contrast, symmetries were \emph{semantic}, when it is not inherent to the encoding.
To find symmetries in SAT problem, the formula is transformed into colored graph
and an automorphism tool is applied onto. Specifically, given a formula $\varphi$ with
$m$ clauses over $n$ variables, the graph is constructed as follows:
\begin{itemize}
	\item \emph{clause nodes}: represent each of the $m$ clauses by a node with color 0;
	\item \emph{literals nodes}: represent each of the $l$ literals by a node with color 1;
	\item \emph{clauses edges}: connect each clause node to the node of the literals that appear in clause;
	\item \emph{boolean consistency edges}: connect each pair of literals that correspond to the same variables.
\end{itemize}


\begin{figure}[h]
	\begin{minipage}[c]{.2\textwidth}
		\lstinputlisting[numbers=none]{cnfs/battleship-3-4-unsat.cnf}
	\end{minipage}
	\begin{minipage}[l]{.75\textwidth}
		\includegraphics[width=4.3in]{cnfs/graph_cnf_no_opt-crop}
	\end{minipage}
\caption{Example of constructed symmetry graph for a given CNF}
\end{figure}

\hakan{Explication du graph + informations num nodes num edges. Probleme reel battleship
}


\hakan{extension of permutation to clauses assignments ...}

An optimization of this graph is possible with the usage of binary clauses i.e. a clause with only two literals.
The clause node can be omitted and we connect the two literals. As we cannot distinguish between the optimized edge 
and boolean consistency edges, we must check if the produced permutations are spurious. 
To do so, as we ensure the permutation commutes with the negation it suffice to check:
$\forall x \in Supp(g), g.\neg x = \neg g.x$.
Roughly speaking, we check if the image of the negation of $x$ is equals to the negation of the image of $x$,
for each element $x$ in the support of the permutation.
In the previous example, the number of nodes in the graph is reduced to \hakan{COUNT NODES}.

\begin{figure}[h]
	\begin{minipage}[c]{.2\textwidth}
		\lstinputlisting[numbers=none]{cnfs/battleship-3-4-unsat.cnf}
	\end{minipage}
	\begin{minipage}[l]{.75\textwidth}
		\includegraphics[width=4.3in]{cnfs/graph_cnf_opt-crop}
	\end{minipage}
	\caption{Example of constructed symmetry graph for a given CNF}
\end{figure}


\hakan{optimisation du graph}\\
\hakan{creation d'un probleme fil conducteur and utilisation de celui dans chaque partie, du calcul des symétries
jusqu'au SBP}\\

When these graph is given to an automorphism tool like \bliss, the following \emph{generators} are 
obtained:\\
$g_0$: (2 3)(5 6)(8 9)(11 12)(-2 -3)(-5 -6)(-8 -9)(-11 -12)\\
$g_1$: (4 5 6)(7 9 8)(-4 -5 -6)(-7 -9 -8)\\
$g_2$: (4 7)(5 8)(6 9)(-4 -7)(-5 -8)(-6 -9)\\
$g_3$: (1 2)(5 6)(7 9)(10 11)(-1 -2)(-5 -6)(-7 -9)(-10 -11)\\
$g_4$: (1 10)(2 11)(3 12)(-1 -10)(-2 -11)(-3 -12)\\

 
 
 \begin{figure}[h]
 	\centering
 		\includegraphics[width=\textwidth]{cnfs/orbits}
 	\caption{Orbits}
 \end{figure}



\section{Usage of symmetries}

Since the set of permutations is obtained from the formula

We say that $g$ is a symmetry of $\varphi$ if $\alpha \models \varphi$ and $ g.\alpha \models \alpha$


The optimal approach to solve a symmetric SAT problem would be to explore
only one assignment per orbit (for instance each lex-leader). However, finding the
lex-leader of an orbit is computationally hard\hakan{CITATION}. 
However, some practical algorithms exists to solve this problem.


\emph{Symmetry breaking} aims at eliminating symmetry, either
by \emph{statically} posting symmetry breaking constraints that invalidate symmetric
assignments, or by altering the search space \emph{dynamically} to avoid symmetric search paths.

\emph{static symmetry breaking} acts like a preprocessor which add \emph{symmetry breaking predicates} (SBP)
at the original formula and solve the augmented problem.


\subsection{Lexicographic leader}
\subsection{Compact CNF encoding of Lexicographic leader}

\subsection{Some special groups}




%% Local Variables:
%% TeX-master: "main.tex"
%% ispell-dictionary: "en_US"
%% mode: latex
%% mode: flyspell
%% coding: utf-8
%% End:
