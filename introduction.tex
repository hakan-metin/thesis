\chapter{Introduction}\label{chap:intro}

Nowadays, computers are powerful and used in many applications in different domains.
One of these domains is critical application that run in planes, cars, trains, etc.
Software present in these machines must be correct and exempt of bugs.
Proving the correctness of these software is a difficult problem. 



 can leads to combinatorial explosion.
Over the years, computer scientists have developed many techniques to solve 
these kinds of problems like \emph{constraint programming} (CP) \cite{rossi2006handbook},
\emph{Propositional Satisfiability} (SAT) \cite{biere2009handbook},
\emph{Satisfiability Modulo Theory} (SMT) \cite{barrett2018satisfiability}.
In this thesis, we focus on solving propositional or Boolean formula, it consists of deciding whether 
all constraints of a formula can be satisfiable and answer \sat, or it is not possible to satisfy all
constraints and answer \unsat.This problem may appear to be simple but cannot be handled efficiently
at the moment. This is due to the complexity of the problem which is proved NP-complete in 1971. 
Many industrial applications problems can be transformed into a SAT problem.
Improving the performance of tools that resolve this problem is an important challenge. 

Over the last decades, SAT solver can handle more and more complicated problems in different domains:
like \emph{formal methods}: hardware model checking,
software model checking, etc.; \emph{artificial intelligence}: planning~\cite{planning_92}; \emph{game resolution}:
sudoku, n-queens, \emph{Bioinformatics} : Haplotype inference,
\emph{design automation} : equivalence checking.
A recent work solve the Pythagorean triple, an old mathematical problem which has been resolved with 
a SAT solver and produce a huge proof of 200 TB.
 
This success comes from the introduction of sophisticated heuristics and optimization of the solving 
algorithm called Conflict Driven Clause Learning (CDCL) algorithm. It is based on the first non memory
intensive algorithm named by its authors Davis, Putnam, Logemann, and Loveland (DPLL).
Unfortunately, some problems still intractable for state-of-the-art SAT solvers. But some 
of them exhibits symmetries that can be exploited by the solver to accelerate the overall solving time.
At its most basic, symmetry is some transformations of an object that leaves it unchanged.
Symmetries is common in real life, if we take some butterfly, it has exactly the same halves.
In the case of satisfiability problems, it maps a solution of a problem to another.
Ignoring these properties forces the solver to explore equivalent search space and it is a loss of
time and energy. Considering the butterfly example if we search a pattern and it is not present, it is
completely absurd to verify the other side. 

In the literature, some works exist and tackle the symmetry problems.
But, the first step to exploit symmetries is to find them. For this purpose, it exists technique 
that use graph isomorphism.
When symmetries are found, the most common approach to exploit it is \emph{static symmetry breaking}.
It takes the symmetric problem as input and produces a satisfiability equivalent formula without symmetries. This transformation is made without any change of existing solvers. This approach works 
well on many applications but stuck on some highly symmetrical ones.

Another approach to handle symmetry is \emph{dynamic symmetry breaking}, it is included in the solver. It observes his behavior and use symmetry properties to avoid visiting symmetric search space.
These approaches will be clearly explained all along this thesis.


\section{Contributions}

Understanding state of the art techniques in symmetry breaking allow us to improve it.
2 majors contributions are detailed in this thesis. The first one use the force of static symmetry 
breaking and apply it dynamically to avoid its drawback. It add an opportunistic symmetry controller 
that avoid visiting symmetric search space. This idea allow us to solve very hard symmetric problems.
 The second contribution use the previous one and combines it with state of the art dynamic 
 symmetry breaking approach and take the best of 2 worlds. This combination leads to 
 important theoretical step for the usage of \emph{partial symmetry breaking} with the usage of 
 \emph{local symmetries}. 
 

\section{Structure of the manuscript}



%% Local Variables:
%% TeX-master: "main.tex"
%% ispell-dictionary: "en_US"
%% mode: latex
%% mode: flyspell
%% coding: utf-8
%% End: