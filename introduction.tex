\chapter{Introduction}\label{chap:intro}



Propositional logic deals with propositions which can be true or false.
The logical connection allows the proposals to be connected to each other.
In this thesis, we focus on solving the problem of Boolean satisfiability (SAT).

Given a propositional formula that are generally the constraints of an encoded problem,
SAT solving consists in deciding whether the formula is satisfiable (i.e, all constraints can be
satisfiable) or unsatisfiable (i.e, there is no way to satisfy all constraints).
This computation is made by a SAT solver that answer $\sat$ when the formula is satisfiable
and $\unsat$ otherwise.

This problem may appear to be simple but cannot be handled efficiently
at the moment. This is due to the complexity of the problem which is prove to be NP-complete in 1971~\cite{cook1971complexity}. Many industrial applications problems can be transformed into a SAT problem.
Improving the performance of SAT solver is an important challenge.


An example of problem can be the vehicle routing problem (VRP).
It concerns the service of delivery company and deals with the following question:
How to deliver things from depots to a set of customers with a fleet of vehicle that can move on road network.
The goal is to find a set of routes that minimize the delivery cost (monetary, distance, ...).
This problem quickly becomes unachievable for a human being. Moreover, 
finding the optimal solution for VRP problem is NP-Hard~\cite{toth2002vehicle}.


This problem leads to combinatorial explosion of the search space.
It greatly hinders the performance of SAT solvers.
Different approach have been proposed to to deal with this combinatorial explosion.
Here, we focus on the exploitation of \textit{symmetries}. 


\section{Symmetry}

In general, a symmetry is a transformation that leave an object (or some aspect of the object) unchanged.
Many problem exhibits symmetry, for example in our previous example renaming a set of identical vehicle,
rotating a chess board, 
The presence of symmetry force search algorithm to fruitlessly explore symmetric search space.


Symmetries are typically defined as a \textit{syntactical} property of a problem. 
Its presence is inherent to the encoding of the problem.
A permutation of variable preserve the original specification. 
Conversely, a \textit{semantic} symmetries are independent of any 
particular representation of the problem.


To exploit the symmetry of a problem, the first step is to find them.
In context of SAT, the detection of symmetry is done by a transforming the specification
in a colored graph and then apply a graph automorphism tool.



When symmetries are found, the most common approach to exploit it is \emph{static symmetry breaking}.
It takes the symmetric problem as input and produces a satisfiability equivalent formula without symmetries.
It is done by augmenting the problem with constraints that force solver to not explore the symmetric search 
space. It is a easy way to integrate symmetry breaking in an existing SAT solver, no modification 
of existing solver is necessary. In addition, this approach works well on many symmetry applications.
However, the number and the size of added constraints increase the solving time and the solver stuck
on some highly symmetrical problems.


Another approach to handle symmetry is \emph{dynamic symmetry breaking}. In this one, the management of
symmetries is included in the search algorithm. It observes the behavior of the solver and help it to
not visit symmetric search space.


The challenge of this thesis is to understand the state of  the art approach in symmetry breaking and
improve them.

%all constraints of a formula can be satisfiable and answer ,
% or it is not possible to satisfy all constraints and answer \unsat.
%
%The problem of Boolean satisfiability (SAT) is nowadays an unavoidable problem. 
%
%
%
%SAT solving consists in deciding whether 
%all constraints of a formula can be satisfiable and answer \sat, or it is not possible to satisfy all
%constraints and answer \unsat.
%
%
%It consists of deciding whether 
%all constraints of a formula can be satisfiable and answer \sat, or it is not possible to satisfy all
%constraints and answer \unsat.








Nowadays, computers are powerful and used in many applications in different domains.
One of these domains is critical application that run in planes, cars, trains, etc.
Software present in these machines must be correct and exempt of bugs.
Proving the correctness of these software is a difficult problem. 

 can leads to combinatorial explosion.
 
 
Over the years, computer scientists have developed many techniques to solve 
these kinds of problems like \emph{constraint programming} (CP) \cite{rossi2006handbook},
\emph{Propositional Satisfiability} (SAT) \cite{biere2009handbook},
\emph{Satisfiability Modulo Theory} (SMT) \cite{barrett2018satisfiability}.



In this thesis, we focus on solving propositional or Boolean formula, it consists of deciding whether 
all constraints of a formula can be satisfiable and answer \sat, or it is not possible to satisfy all
constraints and answer \unsat.
This problem may appear to be simple but cannot be handled efficiently
at the moment. This is due to the complexity of the problem which is prove to be NP-complete in 1971. 
Many industrial applications problems can be transformed into a SAT problem.
Improving the performance of tools that resolve this problem is an important challenge. 

Over the last decades, SAT solver can handle more and more complicated problems in different domains:
like \emph{formal methods}: hardware model checking,
software model checking, etc.; \emph{artificial intelligence}: planning~\cite{planning_92}; \emph{game resolution}:
sudoku, n-queens, \emph{Bioinformatics} : Haplotype inference,
\emph{design automation} : equivalence checking.
A recent work solve the Pythagorean triple, an old mathematical problem which has been resolved with 
a SAT solver and produce a huge proof of 200 TB.
 
This success comes from the introduction of sophisticated heuristics and optimization of the solving 
algorithm called Conflict Driven Clause Learning (CDCL) algorithm. It is based on the first non memory
intensive algorithm named by its authors Davis, Putnam, Logemann, and Loveland (DPLL).
Unfortunately, some problems still intractable for state-of-the-art SAT solvers. But some 
of them exhibits symmetries that can be exploited by the solver to accelerate the overall solving time.
At its most basic, symmetry is some transformations of an object that leaves it unchanged.
Symmetries is common in real life, if we take some butterfly, it has exactly the same halves.
In the case of satisfiability problems, it maps a solution of a problem to another.
Ignoring these properties forces the solver to explore equivalent search space and it is a loss of
time and energy. Considering the butterfly example if we search a pattern and it is not present, it is
completely absurd to verify the other side. 

In the literature, some works exist and tackle the symmetry problems.
But, the first step to exploit symmetries is to find them. For this purpose, it exists technique 
that use graph isomorphism.
When symmetries are found, the most common approach to exploit it is \emph{static symmetry breaking}.
It takes the symmetric problem as input and produces a satisfiability equivalent formula without symmetries. This transformation is made without any change of existing solvers. This approach works 
well on many applications but stuck on some highly symmetrical ones.

Another approach to handle symmetry is \emph{dynamic symmetry breaking}, it is included in the solver. It observes his behavior and use symmetry properties to avoid visiting symmetric search space.
These approaches will be clearly explained all along this thesis.


\section{Contributions}

Understanding state of the art techniques in symmetry breaking allow us to improve it.
2 majors contributions are detailed in this thesis. The first one use the force of static symmetry 
breaking and apply it dynamically to avoid its drawback. It add an opportunistic symmetry controller 
that avoid visiting symmetric search space. This idea allow us to solve very hard symmetric problems.
 The second contribution use the previous one and combines it with state of the art dynamic 
 symmetry breaking approach and take the best of 2 worlds. This combination leads to 
 important theoretical step for the usage of \emph{partial symmetry breaking} with the usage of 
 \emph{local symmetries}. 
 

\section{Structure of the manuscript}



%% Local Variables:
%% TeX-master: "main.tex"
%% ispell-dictionary: "en_US"
%% mode: latex
%% mode: flyspell
%% coding: utf-8
%% End: